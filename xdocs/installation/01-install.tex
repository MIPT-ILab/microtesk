%%%%%%%%%%%%%%%%%%%%%%%%%%%%%%%%%%%%%%%%%%%%%%%%%%%%%%%%%%%%%%%%%%%%%%%%%%%%%%%%%%%%%%%%%%%%%%%%%%%

\chapter{Installation}

%%%%%%%%%%%%%%%%%%%%%%%%%%%%%%%%%%%%%%%%%%%%%%%%%%%%%%%%%%%%%%%%%%%%%%%%%%%%%%%%%%%%%%%%%%%%%%%%%%%
\section{System Requirements}

MicroTESK is a set of Java-based utilities that are run from the command line.
It can be used on \textbf{\textit{Windows}}, \textbf{\textit{Linux}} and
\textbf{\textit{OS X}} machines that have \textbf{\textit{JDK 1.7 or later}}
installed. To build MicroTESK from source code or to build the generated
Java models, \textbf{\textit{Apache Ant version 1.8 or later}} is required.
To generate test data based on constraints, MicroTESK needs
the \textbf{\textit{Microsoft Research Z3}} or \textbf{\textit{CVC4}} solver that
can work on the corresponding operating system.

%%%%%%%%%%%%%%%%%%%%%%%%%%%%%%%%%%%%%%%%%%%%%%%%%%%%%%%%%%%%%%%%%%%%%%%%%%%%%%%%%%%%%%%%%%%%%%%%%%%
\section{Installation Steps}

To install MicroTESK, the following steps should be performed:

\begin{enumerate}
  \item Download from \url{http://forge.ispras.ru/projects/microtesk/files} and unpack
        the MicroTESK installation package (the \texttt{.tar.gz} file, latest release) to your
        computer. The folder to which it was unpacked will be further referred to as
        the installation directory (\texttt{<installation dir>}).

  \item Declare the \texttt{MICROTESK{\_}HOME} environment variable and set its value to the path to
        the installation directory (see the \hyperref[Setting_Environment_Variables]
        {Setting Environment Variables} section).

  \item Set the \texttt{<installation dir>/bin} folder as the working directory (add the path to
        the \texttt{PATH} environment variable) to be able to run MicroTESK utilities from any path.

  \item Note: Required for constraint-based generation. Download and install constraint
        solver tools to the \texttt{<installation dir>/tools} directory (see the
        \hyperref[Installing_Constraint_Solvers]{Installing Constraint Solvers} section).
\end{enumerate}

%%%%%%%%%%%%%%%%%%%%%%%%%%%%%%%%%%%%%%%%%%%%%%%%%%%%%%%%%%%%%%%%%%%%%%%%%%%%%%%%%%%%%%%%%%%%%%%%%%%
\subsection{Setting Environment Variables}
\label{Setting_Environment_Variables}

\paragraph{Windows}

\begin{enumerate}
  \item Open the \texttt{System Properties} window.
  \item Switch to the \texttt{Advanced} tab.
  \item Click on \texttt{Environment Variables}.
  \item Click \texttt{New..} under \texttt{System Variables}.
  \item In the \texttt{New System Variable} dialog, specify variable name as
        \texttt{MICROTESK{\_}HOME} and variable value as \texttt{<installation dir>}.
  \item Click \texttt{OK} on all open windows.
  \item Reopen the command prompt window.
\end{enumerate}

\paragraph{Linux and OS X} ~\\

Add the command below to the \texttt{\textasciitilde{}.bash{\_}profile} file
(\textbf{\textit{Linux}}) or the \texttt{\textasciitilde{}/.profile} file (\textbf{\textit{OS X}}):

\begin{lstlisting}[language=bash]
export MICROTESK_HOME=<installation dir>
\end{lstlisting}

To start editing the file, type \texttt{vi \textasciitilde{}/.bash{\_}profile}
(or \texttt{vi \textasciitilde{}/.profile}).
Changes will be applied after restarting the command-line terminal or reboot. You can also
run the command in your command-line terminal to make temporary changes.

%%%%%%%%%%%%%%%%%%%%%%%%%%%%%%%%%%%%%%%%%%%%%%%%%%%%%%%%%%%%%%%%%%%%%%%%%%%%%%%%%%%%%%%%%%%%%%%%%%%

\subsection{Installing Constraint Solvers}
\label{Installing_Constraint_Solvers}

To generate test data based on constraints, MicroTESK requires external constraint solvers.
The current version supports the \href{https://github.com/z3prover}{Z3} and
\href{http://cvc4.cs.nyu.edu}{CVC4} constraint solvers. Constraint executables should be
downloaded and placed to the \texttt{<installation dir>/tools} directory.

\paragraph{Using Environment Variables} ~\\

If solvers are already installed in another directory, to let MicroTESK find them, the following
environment variables can be used: \texttt{Z3{\_}PATH} and \texttt{CVC4{\_}PATH}. They specify
the paths to the Z3 and CVC4 excutables correspondingly.

\paragraph{Installing Z3}

\begin{itemize}
\item \textbf{\textit{Windows}} users should download Z3 (32 or 64-bit version) from the following
       page:\url{http://z3.codeplex.com/releases} and unpack the archive to the
      \texttt{<installation dir>/tools/z3/windows} directory. Note: the executable file path
      is \texttt{<windows>/z3/bin/z3.exe}.

\item \textbf{\textit{UNIX}} and \textbf{\textit{Linux}} users should use one of the links below
      and and unpack the archive to the \texttt{<installation dir>/tools/z3/unix} directory.
      Note: the executable file path is \texttt{<unix>/z3/bin/z3}.

      \begin{tabular} {| l | r |} \hline
      Debian  x64 & \url{http://z3.codeplex.com/releases/view/101916} \\ \hline
      Ubuntu  x86 & \url{http://z3.codeplex.com/releases/view/101913} \\ \hline
      Ubuntu  x64 & \url{http://z3.codeplex.com/releases/view/101911} \\ \hline
      FreeBSD x64 & \url{http://z3.codeplex.com/releases/view/101907} \\ \hline
      \end{tabular}

\item \textbf{\textit{OS X}} users should download Z3 from
      \url{http://z3.codeplex.com/releases/view/101918} and unpack the archive to the
      \texttt{<installation dir>/z3/osx} directory. Note: the executable file path is
      \texttt{<osx>/z3/bin/z3}.

\end{itemize}

\paragraph{Installing CVC4}

\begin{itemize}
\item \textbf{\textit{Windows}} users should download the latest version of CVC4 binary from
      \url{http://cvc4.cs.nyu.edu/builds/win32-opt/} and save it to the
      \texttt{<installation dir>/tools/cvc4/windows} directory as \texttt{cvc4.exe}.

\item \textbf{\textit{Linux}} users download the latest version of CVC4 binary from
      \url{http://cvc4.cs.nyu.edu/builds/i386-linux-opt/unstable/} (32-bit version) or
      \url{http://cvc4.cs.nyu.edu/builds/x86_64-linux-opt/unstable/} (64-bit version) and
      save it to the \texttt{<installation dir>/tools/cvc4/unix} directory as \texttt{cvc4}.

\item \textbf{\textit{OS X}} users should download the latest version of CVC4 distribution
      package from \url{http://cvc4.cs.nyu.edu/builds/macos/} and install it.
      The CVC4 binary should be copied to \texttt{<installation dir>/tools/cvc4/osx} as
      \texttt{cvc4} or linked to this file name via a symbolic link.

\end{itemize}

%%%%%%%%%%%%%%%%%%%%%%%%%%%%%%%%%%%%%%%%%%%%%%%%%%%%%%%%%%%%%%%%%%%%%%%%%%%%%%%%%%%%%%%%%%%%%%%%%%%
\section{Installation Directory Structure}

The MicroTESK installation directory contains the following subdirectories: \\

\begin{tabular}{ | l | l |}
  \hline
  \textbf{arch} & Microprocessor specifications and test templates \\ \hline
  \textbf{bin}  & Scripts to run modeling and test generation tasks \\ \hline
  \textbf{doc}  & Documentation \\ \hline
  \textbf{etc}  & Configuration files \\ \hline
  \textbf{gen}  & Generated code of microprocessor models \\ \hline
  \textbf{lib}  & JAR files and Ruby scripts to perform modeling and \\
  ~             & test generation tasks \\ \hline
  \textbf{src}  & Source code of MicroTESK \\ \hline
\end{tabular}

%%%%%%%%%%%%%%%%%%%%%%%%%%%%%%%%%%%%%%%%%%%%%%%%%%%%%%%%%%%%%%%%%%%%%%%%%%%%%%%%%%%%%%%%%%%%%%%%%%%
\section{Running}

To generate a Java model of a microprocessor from its nML specification, a user
needs to run the \texttt{compile.sh} script (Unix, Linux, OS X) or the \texttt{compile.bat}
script (Windows). For example, the following command generates a model for the miniMIPS
specification:

\begin{lstlisting}[language=bash]
sh bin/compile.sh arch/minimips/model/minimips.nml
\end{lstlisting}

NOTE: Models for all demo specifications are already built and included in the
MicroTESK distribution package. So a user can start working with MicroTESK from
generating test programs for these models.

To generate a test program, a user needs to use the \texttt{generate.sh} script
(Unix, Linux, OS X) or the \texttt{generate.bat} script (Windows). The scripts
require the following parameters:

\begin{itemize}
\item model name;
\item test template file;
\item target test program source code file.
\end{itemize}

For example, the command below runs the \texttt{euclid.rb} test template for
the miniMIPS model generated by the command from the previous example and saves
the generated test program to an assembler file. The file name is based on values
of the \texttt{--code-file-prefix} and \texttt{--code-file-extension} options.

\begin{lstlisting}[language=bash]
sh bin/generate.sh minimips arch/minimips/templates/euclid.rb
\end{lstlisting}

To specify whether Z3 or CVC4 should be used to solve constraints,
a user needs to specify the \texttt{-s} or \texttt{--solver} command-line option as \texttt{z3}
or \texttt{cvc4} respectively. By default, Z3 will be used. Here is an example:

\begin{lstlisting}[language=bash]
sh bin/generate.sh -s cvc4 minimips arch/minimips/templates/constraint.rb
\end{lstlisting}

More information on command-line options can be found on the
\hyperref[Command_Line_Options]{Command-Line Options} section.

%%%%%%%%%%%%%%%%%%%%%%%%%%%%%%%%%%%%%%%%%%%%%%%%%%%%%%%%%%%%%%%%%%%%%%%%%%%%%%%%%%%%%%%%%%%%%%%%%%%
\section{Command-Line Options}
\label{Command_Line_Options}

MicroTESK works in two modes: \emph{specification translation} and \emph{test generation},
which are enabled with the \texttt{--translate} (used by default) and \texttt{--generate} keys
correspondingly. In addition, the \texttt{--help} key prints information on the command-line format.

The \texttt{--translate} and \texttt{--generate} keys are inserted into the command-line by
\texttt{compile.sh}/\texttt{compile.bat} and \texttt{generate.sh}/\texttt{generate.bat} scripts
correspondingly. Other options should be specified explicitly to customize the behavior of MicroTESK.
Here is the list of options:\\

\begin{tabular}{ | p{4cm} | p{1cm} | p{5cm} | p{2.5cm} |}
  \hline
  \textbf{Full name} & \textbf{Short name} & \textbf{Description} & \textbf{Requires} \\ \hline
  --help & -h & Shows help message & \\ \hline
  --verbose & -v & Enables printing diagnostic messages & \\ \hline
  --translate & -t & Translates formal specifications & \\ \hline
  --generate & -g & Generates test programs & \\ \hline
  --output-dir <arg> & -od & Sets where to place generated files & \\ \hline
  --include <arg> & -i & Sets include files directories & --translate \\ \hline
  --extension-dir <arg> & -ed & Sets directory that stores user-defined Java code & --translate \\ \hline
  --random-seed <arg> & -rs & Sets seed for randomizer & --generate \\ \hline
  --solver <arg> & -s & Sets constraint solver engine to be used & --generate \\ \hline
  --branch-exec-limit <arg> & -bel & Sets the limit on control transfers to detect endless loops & --generate \\ \hline
  --solver-debug & -sd & Enables debug mode for SMT solvers & --generate \\ \hline
  --tarmac-log & -tl & Saves simulator log in Tarmac format & --generate \\ \hline
  --self-checks & -sc & Inserts self-checking code into test programs & --generate \\ \hline
  --default-test-data & -dtd & Enables generation of default test data & --generate \\ \hline
  --arch-dirs <arg> & -ad & Home directories for tested architectures & --generate \\ \hline
  --rate-limit <arg> & -rl & Generation rate limit, causes error when broken & --generate \\ \hline
  --code-file-extension <arg> & -cfe & The output file extension & --generate \\ \hline
  --code-file-prefix <arg> & -cfp & The output file prefix (file names are as follows prefix{\_}xxxx.ext, where xxxx is a 4-digit decimal number) & --generate \\ \hline
  --data-file-extension <arg> & -dfe & The data file extension & --generate \\ \hline
  --data-file-prefix <arg> & -dfp & The data file prefix & --generate \\ \hline
\end{tabular}

\begin{tabular}{ | p{4cm} | p{1cm} | p{5cm} | p{2.5cm} |}
  \hline
  --exception-file-prefix <arg> & -efp & The exception handler file prefix & --generate \\ \hline
  --program-length-limit <arg> & -pll & The maximum number of instructions in output programs & --generate \\ \hline
  --trace-length-limit <arg> & -tll & The maximum length of execution traces of output programs & --generate \\ \hline
  --comments-enabled & -ce & Enables printing comments; if not specified no comments are printed & --generate \\ \hline
  --comments-debug & -cd & Enables printing detailed comments; must be used together with --comments-enabled  & --generate \\ \hline
  --no-simulation & -ns & Disables simulation of generated test programs on the model & --generate \\ \hline
  --time-statistics & -ts & Enables printing time statistics & --generate \\ \hline
\end{tabular}

%%%%%%%%%%%%%%%%%%%%%%%%%%%%%%%%%%%%%%%%%%%%%%%%%%%%%%%%%%%%%%%%%%%%%%%%%%%%%%%%%%%%%%%%%%%%%%%%%%%
\section{Settings File}

Default values of options are stored in the \texttt{<MICROTESK{\_}HOME>/etc/settings.xml}
configururation file that has the following format:

\begin{lstlisting}[language=ruby]
<?xml version="1.0" encoding="utf-8"?>
<settings>
  <setting name="random-seed" value="0"/>
  <setting name="branch-exec-limit" value="1000"/>
  <setting name="code-file-extension" value="asm"/>
  <setting name="code-file-prefix" value="test"/>
  <setting name="data-file-extension" value="dat"/>
  <setting name="data-file-prefix" value="test"/>
  <setting name="exception-file-prefix" value="test_except"/>
  <setting name="program-length-limit" value="1000"/>
  <setting name="trace-length-limit" value="1000"/>
  <setting name="comments-enabled" value=""/>
  <setting name="comments-debug" value=""/>
  <setting name="default-test-data" value=""/>
  <setting
    name="arch-dirs" 
    value="cpu=arch/demo/cpu/settings.xml:minimips=arch/minimips/settings.xml"
  />
</settings>
\end{lstlisting}
