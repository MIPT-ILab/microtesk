\chapter{Usage}

%%%%%%%%%%%%%%%%%%%%%%%%%%%%%%%%%%%%%%%%%%%%%%%%%%%%%%%%%%%%%%%%%%%%%%%%%%%%%%%%%%%%%%%%%%%%%%%%%%%
\section{ISA Model Generation}

To generate a Java model of a microprocessor from its nML specification, a user
needs to run the \texttt{compile.sh} script (Unix, Linux, OS X) or the \texttt{compile.bat}
script (Windows). For example, the following command generates a model for the miniMIPS
specification:

\begin{lstlisting}[language=bash]
sh bin/compile.sh arch/minimips/model/minimips.nml
\end{lstlisting}

NOTE: Models for all demo specifications are already built and included in the
MicroTESK distribution package. So a user can start working with MicroTESK from
generating test programs for these models.

%%%%%%%%%%%%%%%%%%%%%%%%%%%%%%%%%%%%%%%%%%%%%%%%%%%%%%%%%%%%%%%%%%%%%%%%%%%%%%%%%%%%%%%%%%%%%%%%%%%
\section{Test Program Generation}

To generate a test program, a user needs to use the \texttt{generate.sh} script
(Unix, Linux, OS X) or the \texttt{generate.bat} script (Windows). The scripts
require the following parameters:

\begin{itemize}
\item model name;
\item test template file;
\item target test program source code file.
\end{itemize}

For example, the command below runs the \texttt{euclid.rb} test template for
the miniMIPS model generated by the command from the previous example and saves
the generated test program to an assembler file. The file name is based on values
of the \texttt{--code-file-prefix} and \texttt{--code-file-extension} options.

\begin{lstlisting}[language=bash]
sh bin/generate.sh minimips arch/minimips/templates/euclid.rb
\end{lstlisting}

To specify whether Z3 or CVC4 should be used to solve constraints,
a user needs to specify the \texttt{-s} or \texttt{--solver} command-line option as \texttt{z3}
or \texttt{cvc4} respectively. By default, Z3 will be used. Here is an example:

\begin{lstlisting}[language=bash]
sh bin/generate.sh -s cvc4 minimips arch/minimips/templates/constraint.rb
\end{lstlisting}

More information on command-line options can be found on the
\hyperref[Command_Line_Options]{Command-Line Options} section.
