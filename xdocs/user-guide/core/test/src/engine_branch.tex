\section{Branch Engine}

The purpose of the branch engine is to generate test programs that contain \emph{branches}, i.e. instructions that can change execution flow.
There are \emph{unconditional branches}, which always result in branching, and \emph{conditional branches}, which may or may not cause branching, depending on some condition.
Given, an instruction sequence, the branch engine enumerates all possible execution traces of the bounded length or selects random ones.
Enumeration (or selection) is carried out in the depth-first search manner;
when the traverser meets a conditional branch, it assigns a condition value and goes on until it reaches the end of the sequence.
If the length of the execution trace exceeds the given bound, the traverser rolls back to the latest conditional branch and changes the condition value (if possible).
Given an execution trace, the branch engine enforces the specified order of execution by generating so-called \emph{control code} and injecting it into the sequence.

\subsection{Parameters}

\begin{itemize}
\item $branch\_exec\_limit$ is an upper bound for the number of executions of a single branch instruction;
\item $trace\_count\_limit$ is an upper bound for the number of execution traces to be returned.
\end{itemize}

More information on the parameters is given in the `Execution Traces Enumeration' section.

\subsection{Description}

Functioning of the \emph{branch} test engine includes the following steps:

\begin{enumerate}
  \item construction of a \emph{branch structure} of an abstract test sequence;
  \item enumeration of \emph{execution traces} of the branch structure;
  \item concretization of the test sequence for each execution trace:
  \begin{enumerate}
    \item construction of a \emph{control} code;
    \item construction of an \emph{initialization} code.
  \end{enumerate}
\end{enumerate}

Let $D$ be the size of the delay slot for an architecture under scrutiny (e.g., $D = 1$ for MIPS, and $D = 0$ for ARM).

\subsection{Branch Structure Construction}

Step 1 consists in scanning the instructions of the abstract test sequence and constructing the branch structure.

A branch structure is a finite and nonempty sequence of the kind
$$\{(op_i,target_i)\}_(i=0)^N,$$
where $op_i \in \{block, if, goto, slot, end\}$ is a type of the ith element,
and $target_i \in \{0, ... ,N-1\}$ is a target label of the ith element (it is defined for if and goto elements only).

A branch structure satisfies the following constraints:

\begin{itemize}
\item
$op_i = end$ if and only if $i = N$;
\item
if $op_i \in \{if, goto\}$, then $i < N + D$;
\item
$op_i = slot$ if and only if $\exists j \in \{i-D, ..., i-1\} (op_j \in \{if, goto\})$.
\end{itemize}

The following features of the branch structure construction should be emphasized.

An abstract call is classified as if if it is supplied with a situation whose name ends with `if-then', e.g.:

\begin{lstlisting}
beq r1, :label do
  situation('beq-if-then', ...)
end
\end{lstlisting}

An abstract call is classified as goto if it is supplied with a situation whose name ends with `goto', e.g.:

\begin{lstlisting}
b :label do
  situation('b-goto', ...)
end
\end{lstlisting}

\subsection{Execution Traces Enumeration}

To describe the next step, it is worth introducing the following denotations.
If $op_i = slot$, then $branch_i$ denotes the index of the $if$ or $goto$ element that proceeds the ith element.
If $op_i \in \{if, goto\}$, then $branch_i = i$.

Step 2 consists in enumerating execution traces for the constructed branch structure.
Given a branch structure $\{(op_i, target_i)\}_(i=0)^N$, an execution trace is a sequence of the kind $\{(pc_i,cond_i)\}_(i=0)^n$,
where $pc_i$ is an integer, and $cond_i \in \{false, true\}$ satisfies the following constraints:

\begin{itemize}
\item $pc_i = 0$;
\item $pc_i = N$ if and only if $i = n$;
\item if $op_{pc_i} = goto$, then $cond_{pc_i} = true$;
\item if $op_{pc_i} = block$ or\\
      if $op_{pc_i} \in \{if, goto\} \wedge D > 0$ or\\
      if $op_{pc_i} = slot \wedge pc_i < branch_i+ D$:
  \begin{itemize}
  \item $pc_{i+1} = pc_i + 1$;
  \end{itemize}
\item if $op_{pc_i} \in \{if, goto\} \wedge D = 0$ or\\
      if $op_{pc_i} = slot \wedge pc_i = branch_i + D$:
  \begin{itemize}
  \item if $cond_{branch_i} = false$:
    \begin{itemize}
    \item $pc_{i+1} = pc_i + 1$;
    \end{itemize}
  \item if $cond_{branch_i} = true$:
    \begin{itemize}
    \item $pc_{i+1} = target_{branch_i}$;
    \end{itemize}
  \end{itemize}
\end{itemize}

Execution traces are enumerated in a random order.
To guide the trace enumeration process, the following parameters are used:

\begin{itemize}
\item
$branch\_exec\_limit$ specifies the maximum number of occurrences of an if or goto element in an execution trace:
  \begin{itemize}
  \item a trace, where there exists an index $i$, such that $op_i \in \{if, goto\}$ and the number of $i$�s occurrences is greater than $branch\_exec\_limit$, is rejected;
  \item the default value is $1$.
  \end{itemize}
\item
$trace\_count\_limit$ specifies the upper bound for the number of execution traces to be returned:
  \begin{itemize}
  \item if $trace_count_limit \ne -1$ and the number of execution traces (for a given value of the $branch\_exec\_limit parameter$) exceeds $trace\_count\_limit$,
    the engine returns the first $trace\_count\_limit$ traces;
  \item the default value is $-1$ (no limit).
  \end{itemize}
\end{itemize}

\subsection{Test Sequence Concretization}

Step 3 consists in concretizing the test sequence for the given execution trace.
It includes two stages:

\begin{enumerate}
\item construction of control code;
\item construction of initialization code.
\end{enumerate}

Stage 1 consists in injecting a special code, so-called control code, into block and slot elements of the branch structure.
The goal is to guarantee that the test sequence is executed as it is specified in the execution trace.
In other words, the control code changes the registers used by the conditional branches in such a way that every time the instruction is executed,
the calculated condition is equal to the one specified in the execution trace.

The control code construction is based on data streams specified as values of the `stream' parameter of the situations of the conditional branches, e.g. (MIPS):

\begin{lstlisting}
beq r1, :label do
  situation('beq-if-then', :stream => 'stream1')
end
\end{lstlisting}

A data stream can be thought as a pair $\langle data, i \rangle$, where $data$ is an array, and $i$ is an index.
Three code patterns expressed in the terms of the target assembly language are defined for each data stream type:

\begin{itemize}
\item $init = \{ i = 0; \}$;
\item $read(r) = \{ r = data[i]; i = i + 1; \}$;
\item $write(r) = \{ data[i] = r; i = i + 1; \}$.
\end{itemize}

These patterns compose a stream preparator, e.g. (ARM):

\begin{lstlisting}
stream_preparator(:data_source => 'REG',
                  :index_source => 'REG') {
  init {
    adr index_source, start_label
  }

  read {
    ldar data_source, index_source
    add index_source, index_source, 8, 0
  }

  write {
    stlr data_source, index_source
    add index_source, index_source, 8, 0
  }
}
\end{lstlisting}

A data stream instance is declared as follows (ARM):

\begin{lstlisting}
data {
  # Array storing the values of the register
  # used by the conditional branch
  label :branch_data
  dword 0x0, 0x0, 0x0, 0x0, ...
}
...
# Stream  Label          Data  Address  Size
stream   :branch_data,   x0,   x10,     128
\end{lstlisting}

Stage 2 consists in constructing an initialization code. It also uses data streams, but in this case, the init and write patterns are applied instead of read.

\subsection{Requirements}

Currently, the branch test engine imposes the following requirements on test templates:

\begin{enumerate}
\item Each conditional branch instruction should be supplied with a situation whose name ends with `if-then', e.g. (MIPS):

\begin{lstlisting}
beq r1, :label do
  situation('beq-if-then', ...)
end
\end{lstlisting}

\item The stream parameter is obligatory for situations of conditional branch instructions, e.g. (MIPS):

\begin{lstlisting}
beq r1, :label do
  situation('beq-if-then', :stream => 'branch_data')
end
\end{lstlisting}

\item A conditional branch instruction's register should coincide with the $data\_source$ register of the corresponding data stream.
\item $data\_source$ and $index\_source$ registers of data streams should be pairwise different.
\item $index\_source$ registers should not be used inside test templates.
\item Each unconditional branch instruction should be supplied with a situation whose name ends with `goto', e.g. (MIPS):

\begin{lstlisting}
b :label do
  situation('b-goto', ...)
end
\end{lstlisting}

\end{enumerate}

\subsection{Limitations}

\begin{enumerate}
\item Situations of non-branch instructions are ignored.
\end{enumerate}

\subsection{Example}

Here is an example illustrating how the branch test engine works (ARM).

\begin{lstlisting}
class BranchGenerationTemplate < ArmV8BaseTemplate
  def pre
    super

    data {
      org   0
      align 8

      # Array storing the values of the register
      # used by the conditional branches
      label :branch_data_0
      dword 0x0, 0x0, 0x0, 0x0,    0x0, 0x0, 0x0, 0x0,
            0x0, 0x0, 0x0, 0x0,    0x0, 0x0, 0x0, 0x0,
            0x0, 0x0, 0x0, 0x0,    0x0, 0x0, 0x0, 0x0,
            0x0, 0x0, 0x0, 0x0,    0x0, 0x0, 0x0, 0x0

      label :branch_data_1
      dword 0x0, 0x0, 0x0, 0x0,    0x0, 0x0, 0x0, 0x0,
            0x0, 0x0, 0x0, 0x0,    0x0, 0x0, 0x0, 0x0,
            0x0, 0x0, 0x0, 0x0,    0x0, 0x0, 0x0, 0x0,
            0x0, 0x0, 0x0, 0x0,    0x0, 0x0, 0x0, 0x0
    }

    # Data stream preparator
    stream_preparator(:data_source => 'REG',
                      :index_source => 'REG') {
      init {
        adr index_source, start_label
      }

      read {
        ldar data_source, index_source
        add index_source, index_source, 8, 0
      }

      write {
        stlr data_source, index_source
        add index_source, index_source, 8, 0
      }
    }
  end

  def run
    # Data stream instances for the conditional branches
    stream :branch_data_0, x0, x10, 32
    stream :branch_data_1, x1, x11, 32

    # Request to the branch test engine:
    # select one execution trace,
    # where each branch instruction is called at most twice
    sequence(:engine => 'branch',
             :branch_exec_limit => 2,
             :trace_count_limit => 1) {
      label :label0
        nop
        cbnz x0, :label0 do
          situation('cbnz-if-then',
            :stream => 'branch_data_0')
        end
        nop
        cbz  x1, :label1 do
          situation('cbz-if-then',
            :stream => 'branch_data_1')
        end
        nop
        b_imm    :label0 do
          situation('b-goto')
        end
      label :label1
        nop
    }.run 10 # Repeat 10 times
  end
end
\end{lstlisting}

In Step 1, a branch structure is constructed (for ARM, $D = 0$):

~\\
\begin{tabular}{|l|l|}
\hline
$(block, \phi)$	& \texttt{nop}\\
\hline
$(if, 0)$       & \texttt{cbnz x0, :label0}\\
\hline
$(block, \phi)$ & \texttt{nop}\\
\hline
$(if, 6)$       & \texttt{cbz x1,  :label1}\\
\hline
$(block, \phi)$ & \texttt{nop}\\
\hline
$(goto, 0)$     & \texttt{b\_imm    :label0}\\
\hline
$(block, \phi)$ & \texttt{nop}\\
\hline
$(end, \phi)$   & \\
\hline
\end{tabular}
~\\

In step 2, a single execution trace ($trace\_count\_limit = 1$),
such that the number of calls of each branch instruction does not exceed two ($branch\_exec\_limit = 2$),
is selected, e.g., $\{0, 1, 0, 1, 2, 3, 6, 7\}$:

~\\
\begin{tabular}{|l|l|l|}
\hline
$pc_0=0$: & $(block, \phi)$ & \texttt{nop}\\
\hline
$pc_1=1$: & $(if, 0)$       & \texttt{cbnz x0, :label0}\\
\hline
$pc_2=0$: & $(block, \phi)$ & \texttt{nop}\\
\hline
$pc_3=1$: & $(if, 0)$       & \texttt{cbnz x0, :label0}\\
\hline
$pc_4=2$: & $(block, \phi)$ & \texttt{nop}\\
\hline
$pc_5=3$: & $(if, 6)$       & \texttt{cbz x1,  :label1}\\
\hline
$pc_6=6$: & $(block, \phi)$ & \texttt{nop}\\
\hline
$pc_7=7$: & $(end, \phi)$   & \\
\hline
\end{tabular}
~\\

In step 3, the test sequence is concretized -- a control and initialization code is constructed:

\begin{lstlisting}
  // Initialization code (preparation)
  adr x10, branch_data_0    // init(branch_data_0)
  movz x0, #0x938f, LSL #0  // cbnz: true
  stlr x0, [x10, #0]        // write(branch_data_0)
  add x10, x10, #8, LSL #0
  movz x0, #0x0, LSL #0     // cbnz: false
  stlr x0, [x10, #0]        // write(branch_data_0)
  add x10, x10, #8, LSL #0
  adr x10, branch_data_0    // init(branch_data_0)
  movz x1, #0x0, LSL #0     // cbz: true

  // Test sequence (stimulus)
  label0_0000:
  nop
  ldar x0, [x10, #0]        // Control code
  add x10, x10, #8, LSL #0  // read(branch_data_0)
  cbnz x0, label0_0000
  nop
  cbz x1, label1_0000
  nop
  b label0_0000
label1_0000:
  nop
\end{lstlisting}
