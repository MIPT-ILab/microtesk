\chapter{Basic Functions}

%%%%%%%%%%%%%%%%%%%%%%%%%%%%%%%%%%%%%%%%%%%%%%%%%%%%%%%%%%%%%%%%%%%%%%%%%%%%%%%%%%%%%%%%%%%%%%%%%%%

\section{Introduction}

MicroTESK generates test programs on the basis of \emph{test templates} that describe
test programs to be generated in an abstract way. Test templates are created using
special Ruby-based test template description language that derives all Ruby features
and provides special facilities. The language is implemented as a library that
implements facilities for describing test cases. Detailed information on Ruby
features can be found in official documentation \cite{ruby-site, ruby-matz}.

MicroTESK uses the JRuby \cite{jruby-site} interpreter to process test templates.
This allows Ruby libraries to interact with other components of MicroTESK written in Java.

Test templates are processed in two stages:
\begin{enumerate}
\item \label{ttp_stage_1} Ruby code is executed to build the internal representation (a hierarchy
      of Java objects) of the test template.
\item \label{ttp_stage_2} The internal representation is processed with various engines to 
      generate test cases which are then simulated on the reference model and printed to files.
\end{enumerate}

This chapter describes facilities of the test template description language and supported test
generation engines.

%%%%%%%%%%%%%%%%%%%%%%%%%%%%%%%%%%%%%%%%%%%%%%%%%%%%%%%%%%%%%%%%%%%%%%%%%%%%%%%%%%%%%%%%%%%%%%%%%%%

\hyphenation{Template}

\section{Test Template Structure}

A test template is implemented as a class inherited from the \texttt{Template} library class
that provides access to all features of the library. Information on the location of
the \texttt{Template} class is stored in the \texttt{TEMPLATE} environment variable.
Thus, the definition of a test template class looks like this:

\begin{lstlisting}[language=ruby]
require ENV['TEMPLATE']

class MyTemplate < Template
\end{lstlisting}

Test template classes should contain implementations of the following methods:

\begin{enumerate}
  \item \texttt{initialize} (optional) - specifies settings for the given test template;
  \item \texttt{pre} (optional) - specifies the initialization code for test programs;
  \item \texttt{post} (optional) - specifies the finalization code for test programs;
  \item \texttt{run} - specifies the main code of test programs (test cases).
\end{enumerate}

The definitions of optional methods can be skipped. In this case, the default
implementations provided by the parent class will be used. The default implementation
of the \texttt{initialize} method initializes the settings with default values. The default
implementations of the \texttt{pre} and \texttt{post} methods do nothing.

The full interface of a test template looks as follows:

\begin{lstlisting}[language=ruby]
require ENV['TEMPLATE']

class MyTemplate < Template

  def initialize
    super
    # Initialize settings here 
  end

  def pre
    # Place your initialization code here
  end

  def post
    # Place your finalization code here
  end

  def run
    # Place your test problem description here
  end

end
\end{lstlisting}

%%%%%%%%%%%%%%%%%%%%%%%%%%%%%%%%%%%%%%%%%%%%%%%%%%%%%%%%%%%%%%%%%%%%%%%%%%%%%%%%%%%%%%%%%%%%%%%%%%%

\section{Reusing Test Templates}

It is possible to reuse code of existing test templates in other test templates.
To do this, you need to subclass the template you want to reuse instead of the
\texttt{Template} class. For example, the \texttt{MyTemplate} class below reuses code from
the \texttt{MyPrepost} class that provides initialization and finalization code for similar
test templates.

\begin{lstlisting}[language=ruby]
require ENV['TEMPLATE']
require_relative 'MyPrepost'

class MyTemplate < MyPrepost

  def run
    ... 
  end

end
\end{lstlisting}

Another way to reuse code is creating code libraries with methods that can be called
by test templates. A code library is defined as a Ruby module file and has the following
structure:

\begin{lstlisting}[language=ruby]
module MyLibrary

  def method1
    ...
  end

  def method2(arg1, arg2)
    ...
  end

  def method3(arg1, arg2, arg3)
    ...
  end

end
\end{lstlisting}

To be able to use utility methods \texttt{method1}, \texttt{method2} and \texttt{method3}
in a test template, the \texttt{MyLibrary} module must be included in that test template as a mixin.
Once this is done, all methods of the library are available in the test template.
Here is an example:

\begin{lstlisting}[language=ruby]
require ENV['TEMPLATE']
require_relative 'my_library'

class MyTemplate < Template
  include MyLibrary

  def run
    method1
    method2 arg1, arg2
    method3 arg1, arg2, arg3
  end

end
\end{lstlisting}

%%%%%%%%%%%%%%%%%%%%%%%%%%%%%%%%%%%%%%%%%%%%%%%%%%%%%%%%%%%%%%%%%%%%%%%%%%%%%%%%%%%%%%%%%%%%%%%%%%%

\section{Test Template Settings}

\subsection{Managing Text Format}

Test templates use the following settings that set up the format of generated
test programs:

\begin{itemize}
\item \texttt{sl{\_}comment{\_}starts{\_}with} - starting characters for single-line comments;
\item \texttt{ml{\_}comment{\_}starts{\_}with} - starting characters for multi-line comments;
\item \texttt{ml{\_}comment{\_}ends{\_}with} - terminating characters for multi-line comments;
\item \texttt{indent{\_}token} - indentation token;
\item \texttt{separator{\_}token} - token used in separator lines.
\end{itemize}

Here is how these settings are initialized with default values in
the \texttt{Template} class:

\begin{lstlisting}[language=ruby]
@sl_comment_starts_with = "//"
@ml_comment_starts_with = "/*"
@ml_comment_ends_with   = "*/"

@indent_token    = "\t"
@separator_token = "="
\end{lstlisting}

The settings can be overridden in the \texttt{initialize} method of a test template.
For example:

\begin{lstlisting}[language=ruby]
class MyTemplate < Template

  def initialize
    super

    @sl_comment_starts_with = ";" 
    @ml_comment_starts_with = "/="
    @ml_comment_ends_with = "=/"

    @indent_token = "  "
    @separator_token = "*" 
  end
  ...
end
\end{lstlisting}

\subsection{Managing Address Alignment}
\label{managing_address_alignment}

The \texttt{.align n} directive may have different interpretation for different assemblers.
By default, MicroTESK assumes that it aligns an address to the next $2^n$ byte boundary.
If this is not the case, to make MicroTESK correctly interpret it,
the \texttt{alignment{\_}in{\_}bytes} function must be overridden in a test template.
This function returns the number of bytes that correponds to n. The default implementation
of the function looks like this:

\begin{lstlisting}[language=ruby]
#
# By default, align n is interpreted as alignment on 2**n byte border.
# This behavior can be overridden.
#
def alignment_in_bytes(n)
  2 ** n
end
\end{lstlisting}

%%%%%%%%%%%%%%%%%%%%%%%%%%%%%%%%%%%%%%%%%%%%%%%%%%%%%%%%%%%%%%%%%%%%%%%%%%%%%%%%%%%%%%%%%%%%%%%%%%%

\section{Text Printing}

The test template description language provides facilities for printing text messages.
Text messages are printed either into the generated source code or into the simulator log.
Here is the list of functions that print text:

\begin{itemize}
\item \texttt{newline} - adds the new line character into the test program;
\item \texttt{text(format, *args)} - adds text into the test program;
\item \texttt{trace(format, *args)} - prints text into the simulator execution log;
\item \texttt{comment(format, *args)} - adds a comment into the test program;
\item \texttt{start{\_}comment} - starts a multi-line comment;
\item \texttt{end{\_}comment} - ends a multi-line comment.
\end{itemize}

\paragraph{Formatted Printing} ~\\

Functions \texttt{text}, \texttt{trace} and \texttt{comment} print formatted text.
They take a format string and a variable list of arguments that provide data to be
printed.

Supported argument types:
\begin{itemize}
\item constants;
\item locations.
\end{itemize}

To specify locations to be printed (registers, memory), the \texttt{location(name, index)}
function should be used. It takes the name of the memory array and the index of the selected
element.

Supported format characters:
\begin{itemize}
\item \texttt{d} - decimal format;
\item \texttt{x} or \texttt{X} - hexadecimal format (lowercase or uppercase letters);
\item \texttt{s} - decimal format for constants and binary format for locations.
\end{itemize}

For example, the code below prints the \texttt{0xDEADBEEF} value as a constant and as
a value stored in a register using different format characters:

\begin{lstlisting}[language=ruby]
prepare reg(1), 0xDEADBEEF
reg1 = location('GPR', 1)
text 'Constants: dec=%d, hex=0x%X, str=%s', 0xDEADBEEF, 0xDEADBEEF, 0xDEADBEEF
text 'Locations: dec=%d, hex=0x%X, str=%s', reg1, reg1, reg1
\end{lstlisting}

Here is how it will be printed:

\begin{lstlisting}
Constants: dec=3735928559, hex=0xDEADBEEF, str=3735928559
Locations: dec=3735928559, hex=0xDEADBEEF, str=11011110101011011011111011101111
\end{lstlisting}

%%%%%%%%%%%%%%%%%%%%%%%%%%%%%%%%%%%%%%%%%%%%%%%%%%%%%%%%%%%%%%%%%%%%%%%%%%%%%%%%%%%%%%%%%%%%%%%%%%%

\section{Random Distributions}
\label{Random_Distributions}

Many tasks involve selection based on \emph{random distribution}. The test template
language includes constructs to describe ranges of possible values and their weights.
To accomplish this task, the following functions are provided:

\begin{itemize}
\item \texttt{range(attrs)} - creates a range of values and its weight, which are 
      described by the \texttt{:value} and \texttt{:bias} attribures. Values can be
      one of the following types:

      \begin{itemize}
      \item \emph{Single} value;
      \item \emph{Range} of values;
      \item \emph{Array} of values;
      \item \emph{Distribution} of values.
      \end{itemize}

      The \texttt{:bias} attribute can be skipped which means default weight. Default weight
      is used to describe an even distribution based on ranges with equal weights.

\item \texttt{dist(*ranges)} - creates a random distribution from a collection of
       ranges.
\end{itemize}

The code below illustrates how to create weighted distributions for integer numbers:

\begin{lstlisting}[language=ruby]
simple_dist = dist(
  range(:value => 0,          :bias => 25), # Value
  range(:value => 1..2,       :bias => 25), # Range
  range(:value => [3, 5, 7],  :bias => 50)  # Array
  )

composite_dist = dist(
  range(:value=> simple_dist, :bias => 80), # Distribution
  range(:value=> [4, 6, 8],   :bias => 20)  # Array
  )
\end{lstlisting} 

Distributions are used in a number of test template features that will be described
further in this chapter.

%%%%%%%%%%%%%%%%%%%%%%%%%%%%%%%%%%%%%%%%%%%%%%%%%%%%%%%%%%%%%%%%%%%%%%%%%%%%%%%%%%%%%%%%%%%%%%%%%%%

\section{Instruction Calls}

The \texttt{pre}, \texttt{post} and \texttt{run} methods of a test template contain
descriptions of instruction call sequences. Instructions are operations defined
in ISA specifications which represent target assembler instructions. Operations can
have arguments of three kinds:

\begin{itemize}
\item immediate value;
\item addressing mode;
\item operation.
\end{itemize}

Addressing modes encapsulate logic of reading or writing values to memory resources.
For example, an addressing mode can refer to a register, a memory location or hold
an immediate value. Operations are used to describe complex instructions that are
composed of several operations (e.g. VLIW instructions). What arguments are suitable
for specific instructions is specified in ISA specifications.

Arguments are passed to instructions and addressing modes in two ways:
\begin{itemize}
\item As \emph{arrays}. This format is based on methods with a variable number of arguments.
      Values are expected to come in the same order as corresponding parameter definitions
      in specifications. 

\item As \emph{hash maps}. This format implies that operations and addressing modes are
      parameterized with hash tables where the key is in the name of the parameter
      and the value is the value to be assigned to this parameter.
\end{itemize}

The first way is more preferable as it is simpler and closer to the assembly code syntax.
The code below demonstrates both ways (miniMIPS):

\begin{lstlisting}[language=ruby]
# Arrays
add reg(11), reg(9), reg(0)
# Hash maps
add :rd=>reg(:i=>11), :rs=>reg(:i=>9), :rt=>reg(:i=>0)
\end{lstlisting}

\subsection{Aliases}

Sometimes it is required to define \emph{aliases} for addressing modes or operations
invoked with certain arguments. This is needed to make a test template more 
human-readable. This can be done by defining in a test template Ruby functions that
create instances with specific arguments. For example, the following code makes it
possible to address registers \texttt{reg(0)} and \texttt{reg(1)} as
\texttt{zero} and \texttt{at}:

\begin{lstlisting}[language=ruby]
def zero
  reg(0)
end

def at
  reg(1)
end
\end{lstlisting}

\subsection{Pseudo Instructions}

It is possible to specifify \emph{pseudo instructions} that do not have correspondent
operation in specifications. Such instructions print user-specified text and do not
change the state of the reference model. The can be described using the following
function: \texttt{pseudo(text)}. For example:

\begin{lstlisting}[language=ruby]
pseudo 'syscall'
\end{lstlisting}

\subsection{Groups}

Addressing modes and operations can be organized into \emph{groups}. Groups are
used when it is required to randomly select an addressing mode or an operation
from the specified set.

Groups can be defined in specifications or in test templates. To define them in
test templates, the following functions are used: 

\begin{itemize}
\item \texttt{define{\_}mode{\_}group(name, distribution)} - defines an addressing mode\\group;
\item \texttt{define{\_}op{\_}group(name, distribution)} - defined an operation group.
\end{itemize}

Both function take the \texttt{name} and \texttt{distribution} arguments that specify
the group name and the distribution used to select its items.
More information on distributions is in the
\hyperref[Random_Distribution]{Random Distribution} section.
\emph{Notes}: (1) distribution items can be names of addressing modes and operations, by not
names of groups; (2) it is not allowed to redefine existing groups.

For example, the code below creates an instruction group called \texttt{alu} that contains
instructions \texttt{add}, \texttt{sub}, \texttt{and}, \texttt{or}, \texttt{nor}, and \texttt{xor}
selected randomly according to the specified distribution.

\begin{lstlisting}[language=ruby]
alu_dist = dist(
  range(:value => 'add',                       :bias => 40),
  range(:value => 'sub',                       :bias => 30),
  range(:value => ['and', 'or', 'nor', 'xor'], :bias => 30))

define_op_group('alu', alu_dist)
\end{lstlisting}

The following code specifies three calls that use instructions randomly selected
from the \texttt{alu} group:
 
\begin{lstlisting}[language=ruby]
alu t0, t1, t2
alu t3, t4, t5
alu t6, t7, t8
\end{lstlisting}

\subsection{Test Situations}

Test situations are associated with specific instruction calls and specify methods
used to generate their input data. There is a wide range of data generation methods
implemented by various data generation engines. Test situations are specified using the
{\texttt situation} construct. It takes the situation name and a map of optional attributes
that specify situation-specific parameters. For example, the following line of code
causes input registers of the {\texttt add} instruction to be filled with zeros:

\begin{lstlisting}[language=ruby, emph={situation}]
add t1, t2, t3 do situation('zero') end
\end{lstlisting}

When no situation is specified, a default situation is used. This situation places random
values into input registers. It is possible to assign a custom default situation for
individual instructions and instruction groups with the {\texttt set{\_}default{\_}situation} function.
For example:

\begin{lstlisting}[language=ruby, emph={situation, set_default_situation}]
set_default_situation 'add' do situation('zero') end
\end{lstlisting}

Situations can be selected at random. The selection is based on a distribution. This
can be done by using the {\texttt random{\_}situation} construct. For example:

\begin{lstlisting}[language=ruby, emph={situation, random_situation, dist, range}]
sit_dist = dist(
  range(:value => situation('add.overflow')),
  range(:value => situation('add.normal')),
  range(:value => situation('zero')),
  range(:value => situation('random', :dist => int_dist))
  )

add t1, t2, t3 do random_situation(sit_dist) end
\end{lstlisting}

Unknown immediate arguments that should have their va-lues generated are specified using
the "{\texttt \_}" symbol. For example, the code below states that a random value should be added
to a value stored in a random register and the result should be placed to another random register:

\begin{lstlisting}[language=ruby, emph={situation}]
addi reg(_), reg(_), _ do situation('random') end
\end{lstlisting}

\subsection{Registers Selection}

Unknown immediate arguments of addressing modes are a special case and their values are
generated in a slightly different way. Typically, they specify register indexes and are bounded
by the lenght of register arrays. Often such indexes must be selected from a specific range
taking into account previous selections. For example, registers are allocated at random and they
must not overlap. To be able to solve such tasks, all values passed to addressing modes are tracked.
The allowed value range and the method of value selection are specified in configuration files.
Values are selected using the specified method before the instruction call is processed by the
engine that generates data for the test situation. The selection method can be customized by using
the  {\texttt mode{\_}allocator} function. It takes the allocation method name and a map of
method-specific parameters. For example, the following code states that the output register
of the {\texttt add} instruction must be a random register which is not used in the current test case:

\begin{lstlisting}[language=ruby, emph={situation, mode_allocator}]
add reg(_ mode_allocator('free')), t0, t1
\end{lstlisting}

Also, it is possible to exclude some elements from the range by using the {\texttt exclude}
attribute. For example:

\begin{lstlisting}[language=ruby, emph={situation, mode_allocator}]
add reg(_ :exclude=>[1, 5, 7]), t0, t1
\end{lstlisting}

Addressing modes with specific argument values can be marked as free using the
{\texttt free{\_}allocated{\_}mode} function. To free all allocated addressing modes,
the  {\texttt free{\_}all{\_}allocated{\_}modes} function can be used.

%%%%%%%%%%%%%%%%%%%%%%%%%%%%%%%%%%%%%%%%%%%%%%%%%%%%%%%%%%%%%%%%%%%%%%%%%%%%%%%%%%%%%%%%%%%%%%%%%%%

\section{Instruction Call Sequences}

Instruction call sequences are described using block-like structures. Each block specifies
a sequence or a collection of sequences. Blocks can be nested to construct complex sequences.
The algorithm used for sequence construction depends on the type and the attributes of a block.

An individual instruction call is considered a primitive block describing a single sequence
that consists of a single instruction call. A single sequence that consists of multiple calls
can be described using the {\texttt sequence} or the {\texttt atomic} construct. The difference between
the two is that an atomic sequence is never mixed with other instruction calls when sequences
are merged. The code below demonstrates how to specify a sequence of three instruction calls:

\begin{lstlisting}[language=ruby, emph={sequence}, deletekeywords={or}]
sequence {
  add t0, t1, t2
  sub t3, t4, t5
  or  t6, t7, t8
}
\end{lstlisting}

A collection of sequences that are processed one by one can be specified using the {\texttt iterate}
construct. For example, the code below describes three sequences consisting of one instruction call:

\begin{lstlisting}[language=ruby, emph={iterate}, deletekeywords={or}]
iterate {
  add t0, t1, t2
  sub t3, t4, t5
  or  t6, t7, t8
}
\end{lstlisting}

Sequences can be combined using the {\texttt block} construct. The resulting sequences are
constructed by sequentially applying the following engines to sequences returned by
nested blocks:

\begin{itemize}
\item {\texttt combinator} - builds combinations of sequences returned by nested blocks.
      Each combination is a tuple of length equal to the number of nested blocks.
\item {\texttt permutator} - modifies combinations returned by combinator by rearranging
      some sequences.
\item {\texttt compositor} - merges (multiplexes) sequences in a combination into a single
      sequence preserving the initial order of instructions calls in each sequence.
\item {\texttt rearranger} - rearranges sequences constructed by compositor.
\item {\texttt obfuscator} - modifies sequences returned by rearranger by permuting some
      instruction calls.
\end{itemize}

Each engine has several implementations based on different methods. It is possible
to extend the list of supported methods with new implementations. Specific methods
are selected by specifying corresponding block attributes. When they are not specified,
default methods are applied. The format of a block structure for combining sequences
looks as follows:

\begin{lstlisting}[language=ruby, emph={block, iterate, sequence}]
block(
  :combinator => 'combinator-name',
  :permutator => 'permutator-name',
  :compositor => 'compositor-name',
  :rearranger => 'rearranger-name',
  :obfuscator => 'obfuscator-name') {

  # Block A. 3 sequences of length 1: {A11}, {A21}, {A31}
  iterate { A11; A21; A31 }

  # Block B. 2 sequences of length 2: {B11, B12}, {B21, B22}
  iterate { sequence { B11, B12 }; sequence { B21, B22 } }

  # Block C. 1 sequence of length 3: {C11, C12, C13}
  iterate { sequence { C11; C12; C13 } }
}
\end{lstlisting}

The default method names are: {\texttt diagonal} for combinator, {\texttt catenation} for compositor,
and {\texttt trivial} for permutator, rearranger and obfuscator. Such a combination of engines
describes a collection of sequences constructed as a concatenation of sequences returned
by nested blocks. For example, sequences constructed for the block in the above example
will be as follows: \{{\texttt A11}, {\texttt B11}, {\texttt B12}, {\texttt C11}, {\texttt C12}, {\texttt C13}\},
\{{\texttt A21}, {\texttt B21}, {\texttt B22}, {\texttt C11}, {\texttt C12}, {\texttt C13}\} and
\{{\texttt A31}, {\texttt B11}, {\texttt B12}, {\texttt C11}, {\texttt C12}, {\texttt C13}\}


%%%%%%%%%%%%%%%%%%%%%%%%%%%%%%%%%%%%%%%%%%%%%%%%%%%%%%%%%%%%%%%%%%%%%%%%%%%%%%%%%%%%%%%%%%%%%%%%%%%

\section{Data}

\subsection{Configuration}

Defining data requires the use of assembler-specific directives. Information on
these directives is not included in ISA specifications and should be provided in test
templates. It includes textual format of data directives and mappings between nML and
assembler data types used by these directives. Configuration information on data
directives is specified in the \texttt{data{\_}config} block, which is usually placed
in the \texttt{pre} method. Only one such block per a test template is allowed.
Here is an example:

\begin{lstlisting}[language=ruby]
data_config(:text => '.data', :target => 'M') {
  define_type :id => :byte, :text => '.byte', :type => type('card', 8)
  define_type :id => :half, :text => '.half', :type => type('card', 16)
  define_type :id => :word, :text => '.word', :type => type('card', 32)

  define_space :id => :space, :text => '.space', :fillWith => 0
  define_ascii_string :id => :ascii, :text => '.ascii', :zeroTerm => false
  define_ascii_string :id => :asciiz, :text => '.asciiz', :zeroTerm => true
}
\end{lstlisting}

The block takes the following parameters:

\begin{itemize}
  \item \texttt{text} (compulsory) - specifies the keyword that marks the beginning of
        the data section in the generated test program;

  \item \texttt{target} (compulsory) - specifies the memory array defined in the nML
        specification to which data will be placed during simulation;

  \label{base_virtual_address}
  \item \texttt{base{\_}virtual{\_}address} (optional) - specifies the base virtual
        address where data allocation starts. Default value is 0;

  \item \texttt{item{\_}size} (optional) - specifies the size of a memory location unit
        pointed by address. Default value is 8 bits (or 1 byte).
\end{itemize}

To set up particular directives, the language provides special methods that must
be called inside the block. All the methods share two common parameters:
\texttt{id} and \texttt{text}. The first specifies the keyword to be used in a
test template to address the directive and the second specifies how it will be
printed in the test program. The current version of MicroTESK provides the following methods:

\begin{enumerate}
  \item \texttt{define{\_}type} - defines a directive to allocate memory for a data element
        of an nML data type specified by the type parameter;

  \label{define_space}
  \item \texttt{define{\_}space} - defines a directive to allocate memory (one or more
        addressable locations) filled with a default value specified by the
        \texttt{fillWith} parameter;

  \item \texttt{define{\_}ascii{\_}string} - defines a directive to allocate memory for an
        ASCII string terminated or not terminated with zero depending on the
        \texttt{zeroTerm} parameter.
\end{enumerate}

The above example defines the directives \texttt{byte}, \texttt{half}, \texttt{word},
\texttt{ascii} (non-zero terminated string) and \texttt{asciiz} (zero terminated string)
that place data in the memory array \texttt{M} (specified in nML using the \texttt{mem} keyword).
The size of an addressable memory location is 1 byte.

\subsection{Definitions}

Data are defined using the \texttt{data} construct. Data definitions can be added to the test
program source code file or placed into a separate source code file. There are two types of
data definitions:

\begin{itemize}
\item \textbf{Global} - defined in the beginning of a test template and can be used by
      all test cases generated by the test template. Global data definitions can be placed in
      the root of the \texttt{pre} or \texttt{run} methods or methods called from these methods.
      Memory allocation is performed during inital processing of a test template (see
      \hyperref[ttp_stage_1]{stage 1} of template processing).

\item \textbf{Test case level} - defined and used by specific test cases. Such definitions
      can be applied multiple times (e.g. when defined in preparators).
      Memory allocation is performed when a test case is generated (see
      \hyperref[ttp_stage_2]{stage 2} of template processing).
\end{itemize}

The \texttt{data} construct has two optional parameters:

\begin{itemize}
\item \texttt{global} - a boolean value that states that the data definition should be
      treated as global regardless of where it is defined.
\item \texttt{separate{\_}file} - a boolean value that states that the generated data
      definitions should be placed into a separate source code file.
\end{itemize}

\subsubsection{Predefined methods}

Here is the list of methods that can be used in \texttt{data} sections:

\begin{itemize}
\item \texttt{align} - aligns data by the amount n passed as an argument. By default, n means
      $2^n$ bytes. How to change this behaviour see \hyperref[managing_address_alignment]{here}.

\item \texttt{org} - sets data allocation origin. Can be used to increase the allocation address,
      but not to descrease it. Its parameter specifies the origin and can be used in two ways:
      \begin{enumerate}
      \item As \textbf{obsolute} origin. In this case, it is specified as a constant value
            (\texttt{org 0x00001000}) and means an offset from the
            \hyperref[base_virtual_address]{base virtual address}.
      \item As \textbf{relative} origin. In this case, it is specified using a hash map
            (\texttt{org :delta => 0x10}) and means an offset from the latest data allocation.
      \end{enumerate}

\item \texttt{label} - associates the specified label with the current address.  
\end{itemize}

\subsubsection{Configurable methods}

Also, here is the list of runtime methods what has been configured in the \texttt{data{\_}config}
section in the previous example:

\begin{itemize}
\item \texttt{space} - increases the allocation address by the number of bytes specified by
      its argument. The allocated space is filled with the value which has been set up by
      the \hyperref[define_space]{\texttt{define{\_}space}} method.
\item \texttt{byte}, \texttt{half}, \texttt{word}
\item \texttt{ascii}, \texttt{asciiz}
\end{itemize}

Here is an example:

\begin{lstlisting}[language=ruby]
data {
  org 0x00001000

  label :data1
  byte 1, 2, 3, 4

  label :data2
  half 0xDEAD, 0xBEEF

  label :data3
  word 0xDEADBEEF

  label :hello
  ascii  'Hello'

  label :world
  asciiz 'World'

  space 6
}
\end{lstlisting}

In this example, data is placed into memory. Data items are aligned by their size
(1 byte, 2 bytes, 4 bytes). Strings are allocated at the byte border (addressable
unit). For simplicity, in the current version of MicroTESK, memory is allocated
starting from the address 0 (in the memory array of the executable model).

%%%%%%%%%%%%%%%%%%%%%%%%%%%%%%%%%%%%%%%%%%%%%%%%%%%%%%%%%%%%%%%%%%%%%%%%%%%%%%%%%%%%%%%%%%%%%%%%%%%

\section{Preparators}

Preparators describe instruction sequences that place data into registers or memory
accessed via the specified addressing mode. These sequences are inserted into test programs to
set up the initial state of the microprocessor required by test situations. It is possible to
overload preparators for specific cases (value masks, register numbers, etc).
Preparators are defined in the {\texttt pre} method using the {\texttt preparator} construct, which uses
the following parameters describing conditions under which it is applied:

\begin{itemize}
\item {\texttt target} - the name of the target addressing mode;
\item {\texttt mask} (optional) - the mask that should be matched by the value in order for the
       preparator to be selected;
\item {\texttt arguments} (optional) - values of the target addressing mode arguments that should
      be matched in order for the preparator to be selected;
\item {\texttt name} (optional) - the name that identifies the current preparator to resolve ambiguity
      when there are several different preparators that have the same target, mask and arguments.
\end{itemize}

It is possible to define several variants of a preparator which are selected at random according to
the specified distribution. They are described using the {\texttt variant} construct. It has two
optional parameters:

\begin{itemize}
\item {\texttt name} (optional) - identifies the variant to make it possible to explicitly select
      a specific variant;
\item {\texttt bias} - specifies the weight of the variant, can be skipped to set up an even
      distribution.
\end{itemize}

Here is an example of a preparator what places a value into a 32-bit register described by
the {\texttt REG} addressing mode and two its special cases for values equal to {\texttt 0x00000000}
and {\texttt 0xFFFFFFFF}:

\begin{lstlisting}[language=ruby, emph={preparator, variant, data}]
preparator(:target => 'REG') {
  variant(:bias => 25) {
    data {
      label :preparator_data
      word value
    }

    la at, :preparator_data
    lw target, 0, at
  }

  variant(:bias => 75) {
    lui target, value(16, 31)
    ori target, target, value(0, 15)
  }
}

preparator(:target => 'REG', :mask => '00000000') {
  xor target, zero, zero
}

preparator(:target => 'REG', :mask => 'FFFFFFFF') {
  nor target, zero, zero
}
\end{lstlisting}

Code inside the {\texttt preparator} block uses the {\texttt target} and {\texttt value} functions to
access the target addressing mode and the value passed to the preparator.

Also, the {\texttt prepare} function can be used to explicitly insert preparators into test programs.
It can be used to create composite preparators. The function has the following arguments:

\begin{itemize}
\item {\texttt target} - specifies the target addressing mode;
\item {\texttt value} - specifies the value to be written;
\item {\texttt attrs} (optional) - specifies the preparator name and the variant name to select
      a specific preparator.
\end{itemize}

For example, the following line of code places value {\texttt 0xDEADBEEF} into the {\texttt t0} register:

\begin{lstlisting}[language=ruby, emph={prepare}]
prepare t0, 0xDEADBEEF
\end{lstlisting}

%%%%%%%%%%%%%%%%%%%%%%%%%%%%%%%%%%%%%%%%%%%%%%%%%%%%%%%%%%%%%%%%%%%%%%%%%%%%%%%%%%%%%%%%%%%%%%%%%%%

\section{Comparators}

Test programs can include self-checks that check validity of the microprocessor state after
a test case has been executed. These checks are instruction sequences inserted in the end of
test cases which compare values stored in registers with expected values. If the values do not
match control is transferred to a handler that reports an error. Expected values are produced by
the MicroTESK simulator. Self-check are described using the  {\texttt comparator} construct which has
the same features as the {\texttt preparator} construct, but serves a different purpose. Here is
an example of a comparator for 32-bit registers and its special case for value equal to
{\texttt 0x00000000}:

\begin{lstlisting}[language=ruby, emph={comparator, prepare}]
comparator(:target => 'REG') {
  prepare target, value
  bne at, target, :check_failed
  nop
}

comparator(:target => 'REG', :mask => "00000000") {
  bne zero, target, :check_failed
  nop
}
\end{lstlisting}

%%%%%%%%%%%%%%%%%%%%%%%%%%%%%%%%%%%%%%%%%%%%%%%%%%%%%%%%%%%%%%%%%%%%%%%%%%%%%%%%%%%%%%%%%%%%%%%%%%%

\section{Exception Handlers}

Test programs can provide handlers of exceptions that occur during their execution.
Exception handlers are descibed using the {\texttt exception{\_}handler} construct.
This description is also used by the MicroTESK simulator to handle exceptions.
Separate exception handlers are described using the {\texttt section} construct nested into
the {\texttt exception{\_}handler} block. The {\texttt section} function has two arguments: {\texttt org}
that specifies the handler's location in memory and {\texttt exception} that specifies names of
associated exceptions. For example, the code below describes a handler for
the {\texttt IntegerOverflow}, {\texttt SystemCall} and {\texttt Breakpoint} exceptions which resumes
execution from the next instruction:

\begin{lstlisting}[language=ruby, emph={exception_handler, section}]
exception_handler {
  section(:org => 0x380, :exception => ['IntegerOverflow',
                                        'SystemCall',
                                        'Breakpoint']) {
    mfc0 ra, cop0(14)
    addi ra, ra, 4
    jr ra 
    nop
  }
}
\end{lstlisting}
